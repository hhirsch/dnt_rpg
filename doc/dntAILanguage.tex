\documentclass[ letterpaper,12pt]{article}

\usepackage[english]{babel}
\usepackage[T1]{fontenc}
\usepackage{a4wide}
\usepackage{graphicx}
\usepackage{epsfig}

\begin{document}

\title{\textbf{DNT AI Language}}

\author
{
DNTeam
}

\maketitle

\newpage

\tableofcontents

\newpage

\section{Overview}

   The DNT AI Language was created to write AI and mission scripts for the
DccNiTghtmare computer game. 

   Why write a new language instead of using an existed one? Simple: first we
have tried {\it LUA} as a language, but its integration with C++ language and poor
compatibility between versions forced us to abandon the use. In the other hand,
a language implemented at the core of the game engine can be much more optimized
and extensible than a generic one.

\section{Restrictions}

\subsection{Expressions/Lines}
A expression must be at the same line. So,

\begin{verbatim}
a = 25 + 45 + b * (d + e);
\end{verbatim}

{\bf is valid}, but

\begin{verbatim}
a = 25 + 45 + b 
    * (d + e);
\end{verbatim}

{\bf isn't}.

\subsection{Function Parameters}

Function parameters must be variables. So, you can't pass constants to funcions.
For Example:

\begin{verbatim}
   posX = 3500
   posZ = 3200
   moveToPosition(SELF_CHARACTER, posX, posZ)
\end{verbatim}

{\bf will work}, but:

\begin{verbatim}
   moveToPosition(SELF_CHARACTER, 3500, 3200)
\end{verbatim}

{\bf won't}.

\section{Comments}
 
   A comment is a line with its first valid character equal to '\#'. For example:

\begin{verbatim}
   # This script is a test to the DNT script language
   # actually, it don't do anything util.
\end{verbatim}

\section{Separators}

The separators, as its names suggests, separates terms on the language. They are:

\begin{itemize}
\item{Comma: ','}
\item{Semi Colon: ';'}
\item{Space: ' '}
\item{Tab: '/t'}
\item{New Line: '/n'}
\item{Null: '/0'}
\end{itemize}

Any separator is valid at any time. {\it For example:}

\begin{verbatim}
int posX posZ
\end{verbatim}

{\it is the same of:}

\begin{verbatim}
int posX,posZ;
\end{verbatim}

\section{Variables}

\subsection{Declaration}

To declare a variable the expression is:

\begin{verbatim}
type separator name [separator name]
\end{verbatim}

For example, some valid definitions:

\begin{verbatim}
bool control
int posX posZ
int time, value
float dist; mean, sum;
\end{verbatim}

\subsection{Types}

\subsubsection{void}

The void type is as in C/C++: a representation of an empty set of values.

\subsubsection{bool}

The bool type is a boolean, taking values {\it true} or {\it false}.

\subsubsection{int}

The integer type represents a discret number.

\subsubsection{float}

The float type represents a float point number.

\subsubsection{string}

The string type represents a character string. String constants must begin and
end with ``. For Example:

\begin{verbatim}
"This is a string constant!"
\end{verbatim}

\subsubsection{character}

The character type represents a game character, playable or non-playable.

\subsubsection{object}

The object type represents an object in the game. This object can be mapObjects,
weapons, etc.

\subsubsection{map}

The map type represents a game map.

\section{Operations}

An operation must be done with compatible types.

\subsection{Parenthesis}

Parenthesis can be used to group operations, just keeping in mind that any open
paranthesis '(' must have a close ')' one. And {\it vice-versa}.

\subsection{Math Operators}

\begin{itemize}
\item{Addition: '+'}
\item{Subtraction: '-'}
\item{Multiplication: '*'}
\item{Division: '/'}
\item{Modulus: '\%'}
\item{Assign: '='}
\end{itemize}

\subsection{Logical Operators}

\begin{itemize}
\item{Equal: "=="}
\item{Not Equal: "!="}
\item{Lesser: "<"}
\item{Greater: ">"}
\item{Greater Equal: ">="}
\item{Lesser Equal: "<="}
\item{And: "\&\&"}
\item{Or: "||"}
\item{Not: '!'}
\end{itemize}

\section{Flux Control}

\subsection{Script Declaration}

All scripts must have a {\it script()} marking its begining. A {\it script()}
must has an {\it end}.

\begin{verbatim}
script()

   #its a multiple variable definition. Each variable
   #is separated with spaces.
   int a b

   #some maths
   a = 2
   b = 10
   a=b + a
end
\end{verbatim}

\subsection{Conditional}

The conditional flux control is an {\it 'if'}. An {\it 'if'} must have an {\it
end} or {\it else}. An {\it else} must have an {\it end} or {\it else}. {\it
else if} is valid. Examples:

\begin{verbatim}
   if( (c > 40) && (a > 10) && (!variavel) )
      c = 12;
      if(c < 40)
         c = 40
      else
         c = c*2+20
      end
   else
      c = d
   end
\end{verbatim}

\begin{verbatim}
   if(a >= 2)
      b = 1
   else 
   if(a <= 0)
      b = 2
   else
      b = 3;
   end
\end{verbatim}

\subsection{Iterational}

The iterational flux control is an {\it while}. Each {\it while} must have an
{\it end}. Examples:

\begin{verbatim}
   while(count <= 5)
      count = count + 1
      c = c + (count * 2)
   end
\end{verbatim}

\begin{verbatim}
   while(true)
      if(control)
         posX = 3500
         posZ = 3200
         moveToPosition(SELF_CHARACTER, posX, posZ)
         posX = 3600
         posZ = 3200
         moveToPosition(SELF_CHARACTER, posX, posZ)
         control = false
      else
         setIdle(SELF_CHARACTER)
         time = 10
         wait(time)
         control = true
      end
   end
\end{verbatim}


\section{Functions}

At this time, you can't implement functions at the script. This is a feature
that we plan to put latter on the language if we need it.

\subsection{Pending Actions}

   There are two types of functions, those that can be done in a time lesser to
a render cycle, and those that remains for more than one cycle. For example,
verify the current position of the character is done in a time lesser than the
render cycle, but make a character move from its current position to another is 
an operation that requers more than one render cycle to be done.
   For the functions that requires more than one render cycle, the interpreter puts
them at an {\it Action Controller} and wait for the action end (with success or
not) to continue the script interpretation. So this kind of functions {\bf must} be
{\bf boolean} ones.

Next, we'll describe the functions implemented to the script use.

\subsection{Quick Pointers}

The quick pointer are pointers to somethings in game that can be used as a
global variable {\it (but can't have his value directly changed)}.

\subsubsection{SELF\_OBJECT}
{\bf type:} object

The {\it SELF\_OBJECT} is the representation of the object owner of the script
(if one exists). This variable is available when the script is one that
controls some object behaviour.

\subsubsection{SELF\_CHARACTER}
{\bf type:} character

The {\it SELF\_CHARACTER} is the representation of the character owner of the script
(if one exists). This variable is available when the script is one that
controls some character behaviour (in combat or not).

\subsubsection{ACTIVE\_CHARACTER}
{\bf type:} character

The {\it ACTIVE\_CHARACTER} represents the current character controlled by the
user.

\subsubsection{ACTUAL\_MAP}
{\bf type:} map

The {\it ACTUAL\_MAP} represents the current opened map.


\subsection{Imediate Expressions}

\subsubsection{true}
{\bf type:} bool

Represents the true value.

\subsubsection{false}
{\bf type:} bool

Represents the false value.

\subsection{Time}

\subsubsection{wait}
{\bf definition:} bool wait(int s)

The wait function made the script waits for {\it s} seconds before continue to be
interpreted. {\bf Returns:} {\it true} if wait, {\it false} if was interrupted.

{\bf Will generate a Pending Action}

\subsection{Movimentation}

\subsubsection{moveToPosition}
{\bf definition:} bool moveToPosition(character char, int x, int z)

Will move the character {\it char} to the map position {\it x,z}. {\bf Returns:} 
{\it true} if moved, {\it false} if can't move or was interrupted.

{\bf Will generate a Pending Action}

\subsubsection{moveToCharacter}
{\bf definition:} bool moveToCharacter(character char, character tgt)

Will move the character {\it char} near to the character {\it tgt}. {\bf Returns:} 
{\it true} if moved, {\it false} if can't move or was interrupted.

{\bf Will generate a Pending Action}

\subsubsection{moveToObject}
{\bf definition:} bool moveToObject(character char, object obj)

Will move the character {\it char} near to the object {\it obj}. {\bf Returns:} 
{\it true} if moved, {\it false} if can't move or was interrupted.

{\bf Will generate a Pending Action}

\subsection{Character}

\subsubsection{setIdle}
{\bf definition:} void setIdle(character char)

Puts the character {\it char} at idle state.

\subsubsection{getPsycho}
{\bf definition:} int getPsycho(character c)

Gets the psycho state of the character {\it c}. {\bf Returns} psycho state of
the character.

\subsubsection{setPsycho}
{\bf definition:} void setPsycho(character c, int psychoState)

Set the psycho state of the character {\it c} to {\it psychoState}.

\subsubsection{getActualLife}
{\bf definition:} int getActualLife(character c)

Gets the current life points of the character {\it c}. {\bf Returns} current
life points of the character.

\subsubsection{getMaxLife}
{\bf definition:} int getMaxLife(character c)

Gets the max life points of the character {\it c}. {\bf Returns} max
life points of the character.

\subsubsection{characterAtRange}
{\bf definition:} characterAtRange(character ref, character target)

Verify if the character {\it target} is at the range of the character {\it ref}.
{\bf Returns:} true if at range, false if isn't.


\subsection{Inventory}

\subsubsection{inventoryHave}
{\bf definition:} bool inventoryHave(character owner, string objectFile)

Verify if the inventory of character {\it owner} have the object with fileName
{\it objectFile}. {\bf Returns: } true if have the object, false otherwise.

\subsection{Feat}

\subsubsection{featActualQuantity}
{\bf definition:} int featActualQuantity(string featID)

Get current quantity of the feat {\it featID}.

\subsubsection{featCost}
{\bf definition:} int featCost(string featID)

Get current cost to use the feat {\it featID}.

\subsubsection{featQuantityPerDay}
{\bf definition:} int featQuantityPerDay(string featID)

Get the quantity per day of the feat {\it featID}.

\subsubsection{getRandomAttackFeat}
{\bf definition:} string getRandomAttackFeat()

This function get a random attack feat that can be used by
the characterOwner of the script on its defined target.

\subsubsection{getPowerfullAttackFeat}
{\bf definition:} string getPowerfullAttackFeat()

This function get the powerfull attack feat that the characterOwner
of the script can use at its defined target.

\subsubsection{getRandomHealFeat}
{\bf definition:} string getRandomHealFeat()

This function get a random heal feat that can be used by
the characterOwner of the script on its defined target.

\subsubsection{getPowerfullHealFeat}
{\bf definition:} string getPowerfullHealFeat()

This function get the powerfull heal feat that the characterOwner
of the script can use at its defined target.

\subsection{Skill}

\subsubsection{skillPoints}
{\bf definition:} int skillPoints(character c, string skillID)

Get the number of points to the skill {\it skillID} of character {\it c}.

\subsubsection{attModifier}
{\bf definition:} int attModifier(character c, string attID)

Get the Attribute Modifier of attribute {\it attID} of character {\it c}.

\subsection{Fight}

\subsubsection{combatEnter}
{\bf definition:} void combatEnter()

Put the engine at combat mode.

\subsubsection{getNearestEnemy}
{\bf definition:} character getNearestEnemy(character dude)

Get the nearest enemy character from the character {\it dude}


\subsection{Missions}

\subsubsection{missionAdd}
{\bf definition:} void missionAdd(string missionFile)

Add a mission {\it missionFile} to the mission Controller.

\subsubsection{missionComplete}
{\bf definition:} void missionComplete(string missionFile, int completeType)

Complete the mission {\it missionFile} with the {\it completeType} solution.

\subsubsection{missionIsActive}
{\bf definition:} bool missionIsActive(string missionFile)

Verify if the mission is still active {\it (aka: inited and not failed or
completed)}. {\bf Returns: } true if the mission is active.

\subsubsection{missionSetTemp}
{\bf definition:} void missionSetTemp(string missionfile, int tmpNumber, int value)

Set the value of a temporary mission variable {\it tmpNumber}, of mission {\it
missionFile} to {\it value}.


\subsubsection{missionGetTemp}
{\bf definition:} int missionGetTemp(string missionFile, int tmpNumber)

Get the value of the temporary mission variable {\it tmpNumber}, of mission {\it
missionFile};

\subsubsection{missionSetXp}
{\bf definition:} void missionSetXp(string missionFile, int xpValue)

Set the ammount of xp to complete the mission {\it missionFile} to {\it
xpValue}.


\section{Uses}

The DNT scripts have some specific uses, described bellow.

\subsection{NPC Normal Behaviour}

The script can control a NPC behaviour. This type of scripts, usually, are
{\it infinite loop} ones, controlling all the actions the NPC do when not on
battle mode.

For example, a script tha will made a character walk to two positions and stay
idle for 10 seconds.

\begin{verbatim}
# A general movimentation script to test
# the NPC movimentation controlled by scripts.

script()
   bool control
   int posX posZ
   int time

   control = true
   while(true)
      if(control)
         posX = 3500
         posZ = 3200
         moveToPosition(SELF_CHARACTER, posX, posZ)
         posX = 3600
         posZ = 3200
         moveToPosition(SELF_CHARACTER, posX, posZ)
         control = false
      else
         setIdle(SELF_CHARACTER)
         time = 10
         wait(time)
         control = true
      end
   end
end
\end{verbatim}

\subsection{NPC Combat Controller}
The script can control the combat behaviour of the NPC. They are one pass
scripts types, controlling the character {\bf per turn}. For example:

\begin{verbatim}
#An NPC behaviour for a turn on the combat mode
#It's our first test for the NPC combat IA scripts

script()
   int actualLife

   # Get the actual life of the character
   actualLife = getActualLife(SELF_CHARACTER)
   if(actualLife < 10)
      if(haveHealFunction)
         #if can cure itself, cure
         healCharacter(SELF_CHARACTER)
      end
      else
         #if can't cure itself, try to run away
         runAwayFromBattle();
      end
   end

   if(canAttack)
      character dude
      dude = getNearestEnemy(SELF_CHARACTER)
      if(!characterAtRange(dude))
         moveToCharacter(SELF_CHARACTER, dude)
      end

      if(canAttack)
         feat at
         at = getRandomAttackFeat()
         featUseAtCharacter(at, dude)
      end
   end

   if(canMove)
      dude = getNearestEnemy(SELF_CHARACTER)
      if(!characterAtRange(dude))
         moveToCharacter(SELF_CHARACTER, dude)
      end
   end
end
\end{verbatim}

{\it Note that the above script isn't fully implemented on the language yet.}

\subsection{Missions}

The scripts can control a mission. Mission scripts must define itselves the
ammount of XP characters receive by completing them (with {\it missionSetXp}
function). When {\it missionComplete} or {\it missionAbort} is called, the
script is ended. 

For example, a mission that is completed when the character have a {\it Beretta
92} with him. 

\begin{verbatim}
# TUTORIAL - I
# The first tutorial mission

script()

   # create the object fileName, since can't have a literal parameter
   string objFile
   objFile = "weapons/definitions/beretta92.wcc"

   #create the mission filename too
   string missionFile
   missionFile = "missions/tutorial/mission1.dsl"

   int xpValue
   xpValue = 100
   missionSetXp(missionFile, xpValue)

   while(true)
      if(inventoryHave(ACTIVE_CHARACTER,objFile))
         int compType
         compType = 1
         missionComplete(missionFile, compType)
      end
   end

end

\end{verbatim}

\subsection{Triggers}

Scripts can controll some trigger events, for example, when you open a box, a
trigger script can be called, making a trap or other things. This type of
scripts isn't fully implemented yet.

\end{document}


