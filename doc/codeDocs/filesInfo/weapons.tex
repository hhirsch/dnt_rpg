\documentclass[letterpaper,12pt]{article}

\usepackage[english]{babel}
\usepackage[T1]{fontenc}
\usepackage{a4wide}
\usepackage{graphicx}
\usepackage{epsfig}

\begin{document}

\title{Weapons File Description}

\author{
DNTeam
}

\maketitle

\abstract{\begin{center}This document describes the Weapons files Struct
          \end{center}}

\newpage

\tableofcontents

\newpage

\section{Introduction}

  The weapon file is some kind of header to the cal3D model that represents
the weapon. Also, there are definitions to weapons categories, range types,
size types, weight types and damage types, each one on its respective file,
as described below.

\section{Types File Struct}

The types files declares the types definitions of weapons. Each file have the
same struct:

\begin{verbatim}
NUMBER_OF_DECLARATIONS
INDEX DECLARATION_NAME
\end{verbatim}

The files are:

\begin{itemize}
\item{Categories: ../data/weapons/types/categories.dcl}
\item{Ranges: ../data/weapons/types/ranges.dcl}
\item{Sizes: ../data/weapons/types/sizes.dcl}
\item{Weights: ../data/weapons/types/weights.dcl}
\item{Damages: ../data/weapons/types/damages.dcl}
\end{itemize}

\subsection{Number of Declarations}
{\bf Type: }Integer\\
Describes how many declarations on the file.

\subsection{Index}
{\bf Type: }Integer\\
Describes the index of the declaration. {\bf Must be in range
[0,Number\_Of\_Declarations-1]}

\subsection{Declaration Name}
{\bf Type: }String\\
The internal name of the declaration.

\section{Weapon File Struct}

The order of each definition on the file is not important. 

\subsection{Model}

Defines the cal3d filename, relative to the ./bin directory.\\

{\bf Syntax: }
\begin{verbatim}
cal3d = filename
\end{verbatim}

\subsection{Category}

Defines the category of the weapon. The category must be a valid one 
declared on the categories.dcl file.\\

{\bf Syntax: }
\begin{verbatim}
category = category_name
\end{verbatim}

\subsection{Range}

Defines the range of the weapon. The range type must be a valid one 
declared on the range.dcl file.\\

{\bf Syntax: }
\begin{verbatim}
range_type = range_name
range_value = range_value_integer
\end{verbatim}

\subsection{Size}

Defines the size of the weapon. The size type must be a valid one
declared on the sizes.dcl file.\\

{\bf Syntax: }
\begin{verbatim}
size_type = size_name
\end{verbatim}

\subsection{Weight}

Define the weight of the weapon. The weight type must be a valid one declared
on the weights.dcl file. The weight value is a float.

{\bf Syntax:}
\begin{verbatim}
weight_type = weight_name
weight_value = weight_value_float
\end{verbatim}


\subsection{Damage}

Each weapon can have one or two types of damages. The types must be a valid one
declared on the damages.dcl. If the weapon has only one damage, it must be the
main\_damage.

{\bf Syntax:}
\begin{verbatim}
main_damage_type = damage_name
second_damage_type = damage_name
\end{verbatim}

\subsection{Damage Dice}

The damage dice defines the damage that the weapon can inflict.

{\bf Syntax:}
\begin{verbatim}
damage_dice = number_of_dices*dice_identifier+sum_numberxcritical_multiplier
\end{verbatim}

{\bf Example:}
\begin{verbatim}
damage_dice = 4*d6+2x3
\end{verbatim}

\subsection{Attack Sound}

Each type of damage can have a distinct sound.

{\bf Syntax:}
\begin{verbatim}
main_attack_sound = ogg_file_name
second_attack_sound = ogg_file_name
\end{verbatim}

\subsection{Object Things}
Those things are the same as the object (*.dcc) ones. Take a look a the objects document to see what they are.


\end{document}
