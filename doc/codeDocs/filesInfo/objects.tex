\documentclass[letterpaper,12pt]{article}

\usepackage[english]{babel}
\usepackage[T1]{fontenc}
\usepackage{a4wide}
\usepackage{graphicx}
\usepackage{epsfig}

\begin{document}

\title{Objects File Description (*.dcc)}

\author{
DNTeam
}

\maketitle

\abstract
{
   \begin{center}
      This document describes the File Struct of Objects (*.dcc)
   \end{center}
}

\newpage

\tableofcontents

\newpage


\section{Introduction}

The object file is just a header to a cal3d object, defining the not relative
to 3D model attributes, like total life points, 2D image to use on inventories,
etc.

\section{*.dcc Struct}

The struct of the file is the bellow:

\begin{verbatim}
name=Object_Name
cal3d=cal3d_FileName
inventory_sizes=inventory_size_X inventory_size_Y
inventory_texture=2d_inventory_texture_fileName
life_points=total_life_points
fortitude=value
reflex=value
will=value
displacement=value
armature_class=value
size_modifier=value
cost=value
\end{verbatim}

\subsection{Name}
{\bf Type:} String\\
Is the name of the object.

\subsection{cal3d}
{\bf Type:} String\\
Is the filename of the cal3d .cfg object of the object.

\subsection{inventory\_sizes}
{\bf Type:} Two integers\\
Is the size of the object on the inventory, in inventory slots. {\bf 0 0} {\it
for no pickable object}.

\subsection{inventory\_texture}
{\bf Type:} String\\
Is the fileName of the texture used to represent the object on the inventory.

\subsection{life\_points}
{\bf Type:} Integer\\
Is the max number of points the object have.

\subsection{fortitude}
{\bf Type:} Integer\\
Is the fortitude of the object.

\subsection{reflex}
{\bf Type:} Integer\\
Is the reflex of the object.

\subsection{will}
{\bf Type:} Integer\\
Is the will of the object.

\subsection{displacement}
{\bf Type:} Integer\\
Is the displacement of the object, if have one.

\subsection{armature\_class}
{\bf Type:} Integer\\
Is the armature class (AC or CA) of the object.

\subsection{cost}
{\bf Type:} Float\\
Is the cost of the object On the monetary unit of the game.

\subsection{static\_scenery}
{\bf Type:} Integer\\
!= 0 if the object is a static scenery one. This uses all render optimizations
to static sceneries. 

\end{document}
