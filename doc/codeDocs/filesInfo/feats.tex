\documentclass[letterpaper,12pt]{article}

\usepackage[english]{babel}
\usepackage[T1]{fontenc}
\usepackage{a4wide}
\usepackage{graphicx}
\usepackage{epsfig}

\begin{document}

\title{Feats File Description}

\author{
DNTeam
}

\maketitle

\abstract{\begin{center}This document describes the Feats files Struct
          \end{center}}


\section{Introduction}

There are (actually) 4 types of feats on DccNiTghtmare:

\begin{itemize}
\item{{\it Attack or Break Feats}: those that inflicts damage to the target;}
\item{{\it Heal or Fix Feats}: those that heal or fix damage of an object;}
\item{{\it Psycho Feats}: those that changes the psycho/phisical status of a
target;} 
\item{{\it Invocation Feats}: those that invocates some object or animals;}
\end{itemize}

Also, like any DNT things, there is a list indexing all feats defined
on the game.

\section{Feats File}

All feats have the same file struct. The way the struct is interpreted is
dependant from each type of feat. The file struct is:

\begin{verbatim}
Feature Name
Feature Description
Requerided Level
Requerided_Factor_Type Requerided_Factor_Id
Concept_Bonus_Type Concept_Bonus_Id
Concept_Against_Type Concept_Against_Id
Target_Type Target_Id
Base_Dice
Aditional_Dice
Quantity_Per_Day,Aditional_Quantity,Aditional_Levels
Cost_To_Use Action_Type Action_Id
Dependent_Feat_Name Dependence_Ratio,Flag_Used
Dependent_Feat_Name Dependence_Ratio,Flag_Used
Dependent_Feat_Name Dependence_Ratio,Flag_Used
Dependent_Feat_Name Dependence_Ratio,Flag_Used
Dependent_Feat_Name Dependence_Ratio,Flag_Used
\end{verbatim}

\subsection{Feature Name}
{\bf Type:} String;\\
It's the name of the feature.

\subsection{Feature Description}
{\bf Type:} String;\\
It's the description of the feature, in only one line, divided by '|' to change lines on the game ({\it will decided latter}).

\subsection{Requerided Level}
{\bf Type:} Integer;\\
It's the minimal level (or value) requerided to use this feat.

\subsection{Requerided Factor Type}
{\bf Type:} String (without spaces);\\
It's the type of the requerided factor (CLASS, ATTRIBUTE, RACE, etc).

\subsection{Requerided Factor Id}
{\bf Type:} String (without spaces);\\
It's the ID of the requerided factor ().


\section{Feats List}

The list of feats is like any other DNT descriptions list, as bellow:

\end{document}

