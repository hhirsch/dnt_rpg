\documentclass[letterpaper,12pt]{article}

\usepackage[english]{babel}
\usepackage[T1]{fontenc}
\usepackage{a4wide}
\usepackage{graphicx}
\usepackage{epsfig}

\begin{document}

\title{Feats File Description}

\author{
DNTeam
}

\maketitle

\abstract{\begin{center}This document describes the Feats files Struct
          \end{center}}

\newpage

\tableofcontents

\newpage


\section{Introduction}

There are (actually) 4 types of feats on DccNiTghtmare:

\begin{itemize}
\item{{\it Attack or Break Feats}: those that inflicts damage to the target;}
\item{{\it Heal or Fix Feats}: those that heal or fix damage of an object;}
\item{{\it Psycho Feats}: those that changes the psycho/phisical status of a
target;} 
\item{{\it Invocation Feats}: those that invocates some object or animals;}
\end{itemize}

Also, like any DNT things, there is a list indexing all feats defined
on the game.

\section{Feats File}

All feats have the same file struct. The way the struct is interpreted is
dependant from each type of feat. The file struct is:

\begin{verbatim}
Feature Name
Feature Description
Requerided Level
Requerided_Factor_Type Requerided_Factor_Id
Concept_Bonus_Type Concept_Bonus_Id
Concept_Against_Type Concept_Against_Id
Target_Type Target_Id
Base_Dice
Aditional_Dice
Quantity_Per_Day,Aditional_Quantity,Aditional_Levels
Range_Type Range_Of_Action
Cost_To_Use Action_Type Action_Id
Dependent_Feat_Name Dependence_Ratio,Used_Flag
Dependent_Feat_Name Dependence_Ratio,Used_Flag
Dependent_Feat_Name Dependence_Ratio,Used_Flag
Dependent_Feat_Name Dependence_Ratio,Used_Flag
Dependent_Feat_Name Dependence_Ratio,Used_Flag
\end{verbatim}

\subsection{General}

\subsubsection{Feature Name}
{\bf Type:} String;\\
It's the name of the feature.

\subsubsection{Feature Description}
{\bf Type:} String;\\
It's the description of the feature, in only one line, divided by '|' to change lines on the game ({\it will decided latter}).

\subsection{Requeriments}

\subsubsection{Requerided Level}
{\bf Type:} Integer;\\
It's the minimal level (or value) requerided to use this feat.

\subsubsection{Requerided Factor Type}
{\bf Type:} String (without spaces);\\
It's the type of the requerided factor (CLASS, ATTRIBUTE, RACE, etc).

\subsubsection{Requerided Factor Id}
{\bf Type:} String (without spaces);\\
It's the ID of the requerided factor (BIO, DEXTERITY, HUMAN\_LLAMA, etc ).

For example, a requerided level of 12 on the class Biology Student is represented in the file as:

\begin{verbatim}
12
CLASS BIO
\end{verbatim}

\subsection{Bonus}

\subsubsection{Concept Bonus Type}
{\bf Type:} String (without spaces);\\
It's the concept type that will bonus the feat dices (if have one).

\subsubsection{Concept Bonus Id}
{\bf Type:} String (without spaces);\\
It's the concept id that will bonus the feat dices (if have one).

\subsection{Defenses (Against)}

\subsubsection{Concept Against Type}
{\bf Type:} String (without spaces);\\
It's the concept type that will be taked as defense bonus of the feat dices (if
have one).

\subsubsection{Concept Against Id}
{\bf Type:} String (without spaces);\\
It's the concept id that will be taked as defense bonus of the feat dices (if
have one).

\subsection{Target}

\subsubsection{Target Type}
{\bf Type:} String (without spaces);\\
It's the type of the target of the feats (usually ALL).

\subsubsection{Target Id}
{\bf Type:} String (without spaces);\\
It's the id of the target of the feats (usually ALL).

\subsection{Dices And Quantity}

\subsubsection{Base Dice}
{\bf Type:} N*dX+Y;\\
The base dice of the feat. The N, X and Y are integer numbers. A attack feat
that inflicts 5d6+8 of damage, is represented by:

\begin{verbatim}
5*d6+8
\end{verbatim}

\subsubsection{Additional Dice}
{\bf Type:} N*dX+Y;\\ 
It's the additional dice that the feat will receive each
time the (actualLevel - requeridedLevel) + Aditional Levels is reached. For
example, if additional levels is 2 and requerided level is 15, the character
will receive a new additional dice at levels 17, 19, 21 and so on. 

\subsubsection{Quantity Per Day}
{\bf Type:} Integer;\\
It's the number of times the Character can use this feat per day.

\subsubsection{Aditional Quantity}
{\bf Type:} Integer;\\
It's the quantity bonus received each additional levels. 

\subsubsection{Additional Levels}
{\bf Type:} Integer;\\
The Arithmetic Progression Ratio of additional things bonus.

\subsection{Range}

\subsubsection{Range Type}
{\bf Type:} String (without spaces);\\ 
The feat type of range. Usually RANGE\_AREA or RANGE\_TARGET, the first, will
for feats that affects all things on a area. The second to a feat that only
affect the target.

\subsubsection{Range Of Action}
{\bf Type:} Integer;\\
The Area Affected by the feat (usually if is a RANGE\_AREA feat type).

\subsection{Costs and Dependencies}

Each feat can have a maximun of 5 dependent feats. In the file all the five
need to be described, if not, just put 0 at used flag.

\subsubsection{Cost To Use}
{\bf Type:} Integer;\\
The cost (of quantity) to use the feat.

\subsubsection{Action Type}
{\bf Type:} String (without spaces);\\
The Type of the action of the feat (ACT\_TYPE\_NORMAL, ACT\_TYPE\_FREE,
ACT\_TYPE\_MOVEMENT).

\subsubsection{Action Id}
{\bf Type:} String (without spaces);\\
The Id of the action of the feat (ACT\_ATTACK, ACT\_HEAL, etc).

\subsubsection{Dependent Feat Name}
{\bf Type:} String (without spaces);\\
The Name of the dependent feat. A Dependent feat is a feat that has its
quantity value affected by the use of this feat.

\subsubsection{Dependence Ratio}
{\bf Type:} Real;\\
The ratio of dependece. A 2 ratio is equal to say that the
use one time the feat will affect in 2 the quantity of the dependent feat. A
0.5 is the inverse, use one time the dependent will be the same of using 2
times the current.

\subsubsection{Used Flag}
{\bf Type:} Integer (0 or 1);\\
The Flag if the dependence is used (0) or not (1).

\section{Feats List}

The list of feats is like any other DNT descriptions list, as bellow:

\begin{verbatim}
Number_of_Feats
feat_index feat_file feat_image feat_internal_name
\end{verbatim}

\subsection{Number of Feats}
{\bf Type:} Integer;\\
The total number of feats avaible on the game.

\subsection{Feat Index}
{\bf Type:} Integer;\\
The vector Index of the feat.

\subsection{Feat File}
{\bf Type:} String (without spaces);\\
The filename of the feat, relative to the ./bin folder.

\subsection{Feat Image}
{\bf Type:} String (without spaces);\\
The feat image filename of the feat, relative to the ./bin folder.

\subsection{Feat Internal Name}
{\bf Type:} String (without spaces);\\
The Feat internal name. It's the Feat ID that will be used any time requerided
by the definitions files.

\end{document}

